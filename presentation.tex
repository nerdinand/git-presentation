\documentclass[12pt,a4paper]{beamer}
\usepackage[ngerman]{babel}
\usepackage[utf8x]{inputenc}
\usepackage{ucs}
\usepackage{amsmath}
\usepackage{amsfonts}
\usepackage{amssymb}

% Style
\usetheme{Bergen}
\usecolortheme{beetle}

\setbeamertemplate{footline}[frame number]
\beamertemplatenavigationsymbolsempty
\setbeamercovered{transparent}

\author{Ferdinand Niedermann und Thomas Ritter}
\title{S beschte wos je hets \textbf{git}s!}
\subtitle{Distributed Version Control}
\begin{document}

\maketitle

\begin{frame}{Was bisher so geschah...}{Der Subversion-Workflow}
  \begin{itemize}
    \item Zentrales Repository
    \item ``Commitzwang''
    \item Verbindung zum Repository
    \item teure Branches
    \item langsam
  \end{itemize}
\end{frame}

\begin{frame}{Und bei Git?}{}
  \begin{itemize}
    \item alles lokal
    \item schnell
    \item klein
    \item kein ``Commitzwang'' (da commits lokal)
    \item verteilt
    \item verschiedene Workflows
  \end{itemize}
\end{frame}

\begin{frame}{Und bei Git? (Forts.)}{}
  \begin{itemize}
    \item billige lokale Branches
    \item github / gitorious
    \item jedes Objekt SHA1
    \item ``content not files''
  \end{itemize}
\end{frame}

\begin{frame}{Wie alles begann}{}
  \begin{itemize}
    \item Linux kernel, BitKeeper
    \item Linus Torvalds
    \item Anforderungen: ähnlich wie Bitkeeper, Sicherheit, Effizienz
    \item 2005: Erste Version in 2 Wochen
  \end{itemize}
\end{frame}



\end{document}