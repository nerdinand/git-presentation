\documentclass{beamer}
\usepackage[ngerman]{babel}
\usepackage[utf8x]{inputenc}
\usepackage{ucs}
\usepackage{amsmath}
\usepackage{amsfonts}
\usepackage{amssymb}
\usepackage{pgfpages}
\pgfpagesuselayout{resize to}[a4paper,landscape]

% Style
\usetheme{Bergen}
\usecolortheme{beetle}

\setbeamertemplate{footline}[frame number]
\beamertemplatenavigationsymbolsempty
\setbeamercovered{transparent}

\author{Ferdinand Niedermann und Thomas Ritter}
\title{S beschte wos je hets \textbf{git}s!}
\subtitle{Distributed Version Control}
\begin{document}

\begin{frame}
 \maketitle
\end{frame}



\begin{frame}{Was bisher so geschah...}{Der Subversion-Workflow}

  \begin{columns}

    \column{0.5\textwidth}	  
      \begin{itemize}
    	\item Zentrales Repository
	\item ``Commitzwang''
	\item Verbindung zum Repository
	\item teure Branches
	\item langsam
      \end{itemize}

    \column{0.5\textwidth}
      \begin{figure}
       \includegraphics[width=1\textwidth]{./images/holy-repo.png}
      \end{figure}

  \end{columns}
\end{frame}

\begin{frame}[<+->]{Und bei Git?}{}
  \begin{itemize}
    \item alles lokal
    \item schnell
    \item klein
    \item kein ``Commitzwang'' (da commits lokal)
    \item verteilt
    \item verschiedene Workflows
    \item billige lokale Branches
  \end{itemize}
\end{frame}


\begin{frame}
  \begin{figure}
   \includegraphics[width=1.0\textwidth]{./images/svn-timeline.png}
  \end{figure}

  \begin{figure}
   \includegraphics[width=1.0\textwidth]{./images/git-timeline.png}
  \end{figure}
\end{frame}

\begin{frame}
 \begin{figure}
  \includegraphics[width=1.0\textwidth]{./images/file-status-lifecycle.png}
 \end{figure}

\end{frame}

\begin{frame}[<+->]{Wie alles begann}{}
  \begin{itemize}
    \item Linux kernel, BitKeeper
    \item Linus Torvalds
    \item Anforderungen: ähnlich wie Bitkeeper, Sicherheit, Effizienz
    \item 2005: Erste Version in 2 Wochen
    \item Wer verwendet Git? Git, Perl, Gnome, Qt, Ruby on Rails, Android, Wine, Fedora, Debian, X.org, VLC, ...

  \end{itemize}
\end{frame}

\begin{frame}{Erstes Beispiel}

\end{frame}

\begin{frame}[<+->]{Und sonst so?}{}
  \begin{itemize}
    \item jedes Objekt SHA1
    \item ``content not files''
  \end{itemize}
\end{frame}

\begin{frame}[<+->]{Git für Fortgeschrittene}{}
  \begin{itemize}
    \item git bisect
    \item git fsck / git gc
    \item git format-patch
    \item git tag v1.0
    \item git blame
    \item git rebase
    \item git cherry-pick
  \end{itemize}
\end{frame}

\begin{frame}{Github.com}
\begin{itemize}
    \item ``Social Coding", vergleichbar mit Sourceforge
	\item Feb. 2008
	\item 222000 Coder, über 726000 Repositories
	\item Wichtige Projekte
	\begin{itemize}
	  \item Ruby
	  \item Ruby on Rails
	  \item Perl
	  \item PHP
	  \item verschiedene JavaScript-Projekte
	\end{itemize}
  \end{itemize}
\end{frame}



\end{document}